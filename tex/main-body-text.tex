%!TEX root = sr-microhap-main.tex

\section*{Introduction}

The existing tools aren't really made for our tasks.  We want to do something that
is computationally efficient and defensible, focuses only on a few pre-specified sites per 
locus, applies to short amplicon sequencing and/or RAD, RAPTURE, or ddRAD data, and 
makes use of alignment quality scores.  Our method operates on SAM/BAM alignment files.

One could try using {\sc stacks} but it barfs on indels.
\section*{Methods}
\subsection*{Individual level}
Consider a reference genomic region. 
and assume that from a certain individual there are $N$ reads aligning to the region
and that each of the $N$ reads is 
long enough that it spans the region. In practice,
with currently available technology, the reads will likely be short
(80 to 300 bp) products of sequencing an amplicon or small known region. It may be that the 
reads were obtained from paired-end read sequencing of a short
fragment and the reads have been
``flashed'' (CITATION) together into a single apparent read.

Let there be $S$ variants 
within this region that have been previously discovered and
are to be typed as very short haplotypes.  
The $S$ variants are known by their positions $(v_1,\ldots,v_S)$ in the reference sequence.
We will think of each $v_s$ ($s \in \{1,\ldots,S\}$) as being a single-base 
position. As the region is short enough to be spanned by a single read (or overlapping 
paired-end reads)  $S$ will typically be small, perhaps $<6$. 

Each read from the individual can be summarized into base calls only at
the positions $(v_1,\ldots,v_S)$ of the reference. We denote the base calls at these 
positions from the $i\thh$ read $y^{(i)} = (y^{(i)}_1, \ldots, y^{(i)}_S)$.  Associated with
each of these base calls are base quality scores 
$Q^{(i)} = (Q^{(i)}_1, \ldots, Q^{(i)}_S)$, which are
PHRED scaled estimates (CITE) of the probability that the base call at each position is incorrect.
We will use $q = (-Q/10)^{10}$ to denote the probability that the base call is incorrect; thus,
 we have $q^{(i)} = (q^{(i)}_1, \ldots, q^{(i)}_S)$. 
 
 Let the true (albeit unknown) haplotypes possessed by
 the (diploid) individual at the $S$ positions of the reference
 be denoted by $a = (a_1,\ldots, a_S)$ and $b = (b_1,\ldots,b_S)$.  If the $i\thh$ read arose
 from the haplotype $a$, then we calculate the probability of its occurrence as:
 \[
 p(y^{(i)}|a, q^{(i)}) = \prod_{s=1}^S \biggl[ (1-q^{(i)}_s)\I(y^{(i)}_s = a_s) +  
 q^{(i)}_s\I(y^{(i)}_s \neq a_s)\biggr].
 \]
 Then, summarizing the quality scores over many reads with the same
 sequence at the $S$ sites is straightforward: let $z^{(j)} = (z_1^{(j)},\ldots,z_S^{(j)})$ denote a haplotype 
at the $S$ sites and let $z^{(j)}\{i\}$ denote the set of all reads $i$ such that $y^{(i)} = z^{(j)}$.
 Then, the probability of all observed reads of sequence $z^{(j)}$ given they 
 arose from haplotype $a$ is
 \[
\prod_{i\in z^{(j)}\{i\}} p(y^{(i)}|a, q^{(i)})
 \]
 which can be summarized into
 \[
 \prod_{s=1}^S \biggl[ \biggl(\I(z^{(j)}_s = a_s)\prod_{i\in z^{(j)}\{i\}}1-q^{(i)}_s\biggr)
 \biggl(\I(z^{(j)}_s \neq a_s)\prod_{i\in z^{(j)}\{i\}} q^{(i)}_s\biggr) \biggr].
 \]

Clearly, if there are $J$ unique sequences (at the $S$ positions) observed
amongst the reads for the individual, then for each sequence $z^{(j)}$ 
we can summarize the quality scores by computing and storing the following
quantities:
\begin{eqnarray*}
C_{j,s} & = & \prod_{i\in z^{(j)}\{i\}}1-q^{(i)}_s \\
W_{j,s} & = & \prod_{i\in z^{(j)}\{i\}} q^{(i)}_s,
\end{eqnarray*}
for $j \in \{1,\ldots, J\}$.  
(Incidentally, these stand for ``Correct'' and ``Wrong.'') Hence, $P_{j,a}$,
the probability of observing all the reads of sequence $z^{(j)}$ from an individual
given that those reads all came from haplotype $a$, can be calculated as
\begin{equation}
P_{j,a} =  \prod_{s=1}^S\biggl[ \I(z_s^{(j)} = a_s)C_{j,s} + 
 \I(z_s^{(j)} \neq a_s) W_{j,s} \biggr].
 \label{eq:pja}
\end{equation}
Note that the number of reads for an individual might be quite high (possibly hundreds of
thousands), so it will behoove us to collapse our data in this way.

Equation~\ref{eq:pja} gives us a model for observations on a collection of sequencing reads
from a single haplotype.  Unfortunately, each diploid organism will have two copies of the
genome, and, therefore, may carry two haplotypes that are either both the same, or which are
different.  If we have a collection of $N$ reads that fall into $J$ unique sequence
categories for
which we have computed $C$ and $W$, we must 
compute the probability of observing all those reads given that the true underlying 
haplotypes in the individual are $a$ and $b$.  Unfortunately, we do not know which of the
reads arose from $a$ and which from $b$.  If $a\neq b$ (\ie if the individual is not
homozygous for the haplotypes) then there are $2^N$ possible assignments of reads to the two
different haplotypes.  Summing over the possible assignments is clearly intractable.  So, we
explore reducing the number of possible assignments first by making the following
\begin{description}
\item[Assumption:] all of the reads from an individual of a particular sequence at
the $S$ sites (\ie all $y^{(i)}$ that equal $z^{(j)}$ for a given $j$) arose
from a single one of the haplotypes ($a$ or $b$) within the individual.
\end{description}
Personally, I think that this assumption is reasonable. If it doesn't hold it means to me that
there is a lot of crappy sequencing going on. Note that by disregarding all of the sequence on
a read except for the base calls at the $S$ sites, we are throwing away some information that
might help us diagnose how often the same sequence at the $S$ sites arise from different
heterozygous genotypes; but we aren't going to worry about that right now.

Let $\Aset$ denote the set of all the reads from haplotype $a$ and $\Bset$ all the remaining
reads that came from haplotype $b$.   Note that $\Aset \cap \Bset = \emptyset$. And we will
denote all the $N$ reads by $\Rset$.
Then we have
\[
P(\Rset|a,b, Q) = \sum_{\Aset,\Bset} \prod_{j \in \Aset} P_{j,a} 
\prod_{k \in \Bset} P_{k,b} 
\]


Note that if there are $S$ variants, there will be at most $2^S$ possible read sequences or 
haplotypes.  That means no more than $2^S(2^S + 1) / 2$ possible pairs of haplotypes in an 
individual and $J < 2^S$.  Hmmm....but if $J$ is fairly laref then that sum is going to be
over $2^J$ possibilities.  That isn't looking so good...
%\subsection{Population level}

%\section*{Results}

%\section*{Discussion}

%\section*{Acknowledgments}
%We are grateful to \ldots 